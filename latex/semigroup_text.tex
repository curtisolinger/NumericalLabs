\documentclass{article}
\usepackage{amsmath}
\usepackage{amsfonts}
\usepackage{parskip}
\usepackage{enumitem}

\title{Numerical Semigroup Invariants and the Second Frobenius Number}
\author{Curtis Olinger}

\begin{document}
\maketitle

Outline:
\begin{enumerate}
\item A quick hook on the Frobenuis Coin Problem
\item Short history of Frobenius and the Frobenius number
\item Ask question about using the maximum number of coins to pay for something. A common problem is using up coins...
\item Introduce the Second Frobenius number

\item Generators
\item Generator example
\item Define numerical semigroup

\item Introduce max and min factorization length
\item Show example of min/max fact length for specific element
\item Introduce set of max factorization length k
\item Show graph of max factorization length k
\item Conclusion
\end{enumerate}


The Frobenius Coin Problem


Imagine in the country you live, the currency is only made up of 3 and 5 cent coins. Ignoring the question of why your country has only these two denominations, you might run into the situation of trying to get change for an 8 cent candybar if you only have two 5 cent coins. You would give the store clerk 10 cents, but could they give you back 2 cents? In this system, no.

In a 3 and 5 coin system, there are values of money that you cannot create such as 2, 3, 4, and 7 cents. But then every value of currency after that is obtainable given you have enough 3 and 5 cent coins. A question you may ask is what is the greatest monetary amount that cannot be obtained from any combination of these coins?

This is an example of the Frobenius Coin Problem and the answer to the above question is called the Frobenius Number, named after Ferdinand Georg Frobenius. He was a German mathematician born February 14, 1877, in a suburb of Berlin. Frobenius worked on diverse fields such as elliptic functions, differential equations, number theory, and group theory and taught at the University of Berlin and the ETH Zurich. 

If there are only two denominations of coins \(x\) and \(y\), the Frobenius Number is given by the equation \(xy - x - y \). In our 3 and 5 cent system, the Frobenius number is \(3 \times 5 - 3 - 5 = 7\) as previously observed.

For a three or more coin system, there is no explicit answer nor are there any fast and efficient computational algorithms for computing the answer.


The Second Frobenius Number


Another situation that arises is how to use up all the change that ones collects, usually in a kitchen catch-all drawer. If I know the price of something ahead of time, I might go through the coins in my drawer and try to use up as many of them as I can. If something cost 15 cents and I have a lot of 3 and 5 cents coins, I would take to the store five 3 cent coins instead of only three 5 cent coins, thus reducing the number of overall coins I have at home. 

The factorization length of a value is the number of coins it takes to add up to that value. There are two factorization lengths for the value of 15 cents, 3 and 5. The max factorization length is the larger of these two numbers and the min factorization length is the smaller. The max factorization length of 15 cents is 5 coins. 

What other values have a max factorization length of 5 coins? Well, 17 cents does as the only way to get 17 cents is with four 3 cents coins and one 5 cent coin. Sometimes the max and min factorization length are the same. Are there any other values with a max factorization length of 5? There is one more: 19, as you can get that with three 3 cent coins and two 5 cent coins. But that is it. The number of elements with a max factorization length of five is 3. 

Now thinking more generally, is the number of elements with a max factorization length of k always 3? No, there are only two numbers with a max factorization length of 1, that is specifically three and five cents. Can you have more that 3 elements with a max factorization length of 3? Again, no. For any factorization length you can think of, the number of elements with that length is always less than or equal to three in our 3, 5 coin system.

This raises an interesting question: given an arbitrary coin system with any number of denomination of coins, does the number of values of money you can create with a max factorization length of k always stabilize at some value. The answer is yes, and the number at which it stabilizes is called The Second Frobenius Number. 


Generators

The coins our country's currency system are example of generators as they can ``generate'' all possible value of money given you have enought coins. 

If your generators are 3 and 5, repeated adding of these numbers we result in the following first ten terms:

CODE HERE

We would write,
\[
	\langle 3, 5 \rangle = \{3, 5, 6, 8, 9, 10, \ldots \}
\]
where \( \langle 3, 5 \rangle\) denotes the set generated by 3 and 5. Notice that you cannot obtain 1, 2, 4, and 7.

Mathematically we would write,
\[
	\mathbb{Z^+} \setminus \langle 5, 7 \rangle = \{ 1, 2, 4, 7 \}
\]
where $\mathbb{Z}^+$ is the set of positive integers. We would have the complement of the set of possible monteary values within the set of positive integers is finite. 

This is not always the case. If we take 2 as a generator, we get all the positive even numbers and its complement in the set of positive integers is not finite. Namely the comlement is the positive odd integers. 




Numerical Semigroups


By examining all the values we can create with whatever denominations of coins we have in our currency system, as long as the coins themselves don`t share any common factors, we've have been creating a structure called a numerical semigroup. The coins themselves are called generators since they ``generate'' all possible values of money. 

Formally, a numerical semigroup is a subset of the nonnegative integers with addition as the binary operations (meaning you can add numbers and the result will be in this set) and a finite complement with the positive integers.

Thus 3 and 5 create a numerical semigroup but 2 does not. In fact, all you need from a set of generators to create a numerical semigroup is for the generators to be relative prime, that is, they only share a common factor of 1. 

Next notice that every numerical semigroup has a set of generators. The easiest way to see this is to consider the smallest non-zero element in the semigroup. This element, together with all other elements in the semigroup that are not divisible by it, form a set of generators of the semigroup.

This is because every element in the semigroup can be expressed as a linear combination of the generators with non-negative integer coefficients. Thus, every numerical semigroup does indeed have a set of generators.

While every numerical semigroup contains a set of generators, the inverse that every set of generators creates a numerical semigroup is not always true. The even integers generated by 2 is an example of this. 


Numerical Semigroups are the main field of study for this article and while they might seem simple, there is a great deal to discover about them. To get started, we will lay out some definitions that will allow us to examine numbers, called invariants, that capture information about the structure of the semigroups.

Max and min factorization lengths

Let \( m_1, \ldots, m_n \) be a set of generators for the numerical monoid \( M \). Then, for any monoid element \(m \), the set of factorizations of \(m \in M \) is
\[
	\mathsf{Z}(m) = \left \{ (a_1, \ldots, a_k) \mid m =
\sum_{i=1}^{k} a_im_i \right \}
\]
and the set of factorization lengths of \(m \in M\) to be
\[
	\mathsf{L}(m) = \left\{\sum_{i = 1}^{k} a_i : (a_1, \ldots, a_k) \in \mathsf{Z}(m)\right\}.
\]

For example, in the monoid generated by 2 and 3, the element 6 has the following two factorizations (3, 0) and (0, 2) since
\[
	(3, 0) \cdot (2, 3) = 3 \times 2 + 0 \times 3 = 6
\]
and
\[
	(0, 2) \cdot (2, 3) = 0 \times 2 + 2 \times 3 = 6
\]
where \( \cdot \) is the dot product operation from linear algebra. Thus
\[
	\mathsf{Z}(6) = \left \{ (3, 0), (0, 2) \right \}.
\]
The element 6 then has the set of factorization lengths of
\[
	\mathsf{L}(m) = \{ 3, 2 \}
\]
and it has a maximum factorization length 3 and a minimum factorization length of 2. In the following form, you can enter some generators and examine particular semigroup elements to see the set of factorization and factorization lengths along with the max length and min length for the element.

CODE HERE

Number of elements of a max factorization length of k




















\end{document}